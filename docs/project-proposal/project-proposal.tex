\documentclass[10pt]{article}

%% Language and font encodings
\usepackage[english]{babel}
\usepackage[utf8x]{inputenc}
\usepackage[T1]{fontenc}
\usepackage{url}

%% Sets page size and margins
\usepackage[a4paper, top=2cm, bottom=2cm, left=2cm, right=2cm, marginparwidth=1.75cm]{geometry}

%% Useful packages
\usepackage{graphicx}

\begin{document}
\begin{center}

    \includegraphics[scale=0.5]{assets/logo.png}

    \begin{LARGE}
        \textbf{SCAR: A Blockchain based approach for academic registry}
    \end{LARGE}

    \vspace{.75cm} Projeto e Seminário

    \vspace{.25cm} Licenciatura em Engenharia Informática e Computadores

    \vspace{.75cm}
    \begin{large}
        \begin{tabular}{ c c c }
            Diogo Rodrigues               & Gonçalo Frutuoso              \\
            \texttt{49513@alunos.isel.pt} & \texttt{49495@alunos.isel.pt} \\
        \end{tabular}

        \vspace{.75cm} Orientadores \vspace{0.25cm}

        \begin{tabular}{ c c }
            Cátia Vaz                    & Alexandre Francisco      \\
            \texttt{cvaz@cc.isel.ipl.pt} & \texttt{aplf@tecnico.pt}
        \end{tabular}
    \end{large}

    \vspace{.75cm} \today \vspace{.75cm}

\end{center}

\section{Introdução}

\subsection*{Motivação}

A veracidade e autenticidade dos registros acadêmicos, especialmente certificados e certidões de notas,
têm sido um desafio persistente no campo da educação. A confiabilidade desses documentos é crucial não apenas
para os próprios indivíduos, mas também para instituições de ensino, empregadores e outras entidades que
dependem desses registros para tomar decisões importantes. No entanto, a atual gestão desses registros
ainda é amplamente baseada em métodos tradicionais, como impressões em papel ou documentos digitais simples,
o que os torna vulneráveis à adulteração, falsificação ou perda. A crescente incidência de fraudes
relacionadas a certificados acadêmicos, juntamente com os desafios associados à validação manual desses
documentos, destaca a necessidade urgente de uma solução inovadora e segura.

Na contemporaneidade, embora existam algumas soluções digitais disponíveis para a gestão de registros acadêmicos,
muitas delas carecem de uma abordagem verdadeiramente segura e confiável. Os sistemas de gestão de aprendizagem
(LMS) e os portais de estudantes geralmente oferecem opções para visualizar e baixar certificados e certidões
de notas, porém, esses documentos digitais ainda podem ser facilmente manipulados ou falsificados. Além disso,
a verificação desses registros é muitas vezes um processo manual e demorado, envolvendo a comunicação direta entre
as partes interessadas ou a consulta a bases de dados centralizadas, o que pode comprometer a eficiência
e a integridade do processo.

Diante desse cenário, surge a necessidade premente de uma solução que possa garantir a autenticidade e a
imutabilidade dos registros acadêmicos, ao mesmo tempo em que simplifica e agiliza o processo de validação.
É nesse contexto que nossa proposta de projeto, denominada SCAR (Smart Contracts for Academic Registry),
se destaca como uma abordagem inovadora e eficaz para a gestão de certificados e certidões de notas por
meio da tecnologia blockchain e smart contracts.

\subsection*{Tecnologias}

\subsubsection*{React Native}

React Native, juntamente com a plataforma Expo, emerge como uma das ferramentas mais poderosas e versáteis
para o desenvolvimento de aplicações móveis. Ao unir a eficiência do React.js com a capacidade de compilação
nativa, React Native permite aos desenvolvedores criar aplicações móveis robustos e de alto desempenho
com uma base de código compartilhada. A plataforma Expo, por sua vez, oferece um conjunto de ferramentas
e serviços adicionais que simplificam e aceleram o processo de desenvolvimento, permitindo aos desenvolvedores
concentrarem-se na criação de experiências de usuário excepcionais.

A valorização do tempo investido pelo nosso grupo na aprendizagem dessas tecnologias é inestimável.
Embora o processo de familiarização com novas ferramentas possa inicialmente parecer desafiador,
o retorno do investimento é evidente. A capacidade de desenvolver aplicações móveis de alta qualidade
de forma rápida e eficiente, aproveitando o poder e a flexibilidade do React Native com o suporte
adicional da plataforma Expo, é incomparável. Cada hora dedicada ao aprendizado dessas tecnologias
representa uma oportunidade de aprimorar nossas habilidades e expandir nosso conjunto de ferramentas,
preparando-nos para enfrentar desafios cada vez mais complexos no mundo do desenvolvimento de software.

\subsubsection*{Solidity}

Solidity, a linguagem de programação utilizada para escrever smart contracts na blockchain Ethereum,
o valor do tempo investido pelo nosso grupo é ainda mais evidente. À medida que a adoção da tecnologia
blockchain continua a crescer e se expandir para além do mundo das criptomoedas, o domínio de Solidity se
torna uma habilidade altamente valorizada e procurada. Smart Contracts desempenham um papel fundamental em
uma ampla gama de aplicações distribuidas (dApps), desde finanças descentralizadas (DeFi) até votação eletrônica
e muito mais.

Ao dedicar tempo e esforço ao aprendizado de Solidity, adquiriu-se não apenas uma habilidade técnica essencial,
mas também um posicionamento na vanguarda da inovação tecnológica. O potencial transformador da blockchain e dos
smart contracts é vasto e multifacetado.

\section{Requisitos}

\subsection*{Intervenientes}

A solução SCAR envolve três principais intervenientes: os candidatos, os empregadores e os administradores
do sistema. Cada um desempenha um papel crucial no ecossistema da plataforma.

\subsection*{Papel de cada interveniente e ações que podem realizar}

\begin{itemize}

    \item \textbf{Candidatos:} Os candidatos são a essência da plataforma SCAR. Eles terão a capacidade de registrar-se na
          aplicação móvel, onde poderão carregar e armazenar seus certificados acadêmicos de forma segura na blockchain.
          Além disso, os candidatos poderão visualizar e compartilhar seus certificados com empregadores durante processos
          de recrutamento, proporcionando uma experiência transparente e eficiente.

    \item \textbf{Empregadores:} Os empregadores desempenham um papel fundamental ao solicitar a verificação de certificados
          acadêmicos durante entrevistas de emprego. Ao acessar a plataforma SCAR, os empregadores podem verificar
          instantaneamente a autenticidade dos certificados apresentados pelos candidatos, garantindo assim um processo
          de recrutamento mais confiável e transparente.

    \item \textbf{Administradores de Faculdade:} Os administradores da faculdade são responsáveis por fornecer e validar
     os certificados acadêmicos emitidos pela instituição de ensino. Terão a capacidade de acessar a plataforma SCAR
      para emitir e autenticar os certificados dos candidatos, garantindo assim a integridade e a validade dos documentos.

\end{itemize}

\subsection*{Requisitos obrigatórios}

\begin{itemize}

    \item\textbf{Implementação da aplicação móvel utilizando React Native com a plataforma Expo:} A escolha do React Native
          juntamente com a plataforma Expo proporciona uma abordagem eficaz para o desenvolvimento de aplicações móveis
          multiplataforma. Isso garante uma experiência consistente para os usutilizadores, independentemente do dispositivo
          utilizado, e facilita o processo de desenvolvimento para a equipe de desenvolvimento.

    \item\textbf{Utilização de smart contracts em Solidity para armazenar e validar os certificados acadêmicos na blockchain
              Ethereum:} A utilização de smart contracts em Solidity oferece uma solução segura e confiável para armazenar
          e validar os certificados acadêmicos na blockchain Ethereum. Essa abordagem garante a integridade e a
          imutabilidade dos dados, tornando a plataforma SCAR altamente confiável e resistente a fraudes.

    \item\textbf{Desenvolvimento de funcionalidades para registro, autenticação e armazenamento seguro de certificados na
              blockchain:} O desenvolvimento de funcionalidades robustas para registro, autenticação e armazenamento seguro
          de certificados na blockchain é essencial para garantir a segurança e a confiabilidade da plataforma SCAR.
          Essas funcionalidades devem ser projetadas com foco na usabilidade e na segurança, proporcionando uma
          experiência intuitiva e transparente para os utilizadores.

\end{itemize}

\subsection*{Requisitos opcionais}

\begin{itemize}

    \item\textbf{Integração de funcionalidades adicionais, como notificações em tempo real e compartilhamento de certificados
através de redes sociais:} A integração de funcionalidades adicionais, como notificações em tempo real e
compartilhamento de certificados através de redes sociais, pode melhorar ainda mais a experiência do
utilizador na plataforma SCAR, aumentando o engajamento e a usabilidade.

    \item\textbf{Implementação de um sistema de reputação ou avaliação para candidatos e empregadores:} A implementação de um
sistema de reputação ou avaliação para candidatos e empregadores pode fornecer uma medida adicional de
confiança e transparência na plataforma SCAR. Isso permite que os utilizadores avaliem e compartilhem suas
experiências, promovendo uma comunidade mais colaborativa e confiável.

\end{itemize}

\section{Arquitetura}

\pagebreak

\section{Calendarização}

\begin{table}[!h]
    \centering
    \begin{tabular}{| c | c | c |}
        \hline
        Data de início  & \begin{tabular}{@{}c@{}}Duração \\ (semanas)\end{tabular} & Descrição                                                        \\
        \hline
        16 de Fevereiro & 1                                                         & Estudo e escolha dos componentes a usar.                         \\
        \hline
        16 de Fevereiro & 3                                                         & Aprendizagem e estudo das tecnologias a serem utilizadas.        \\
        \hline
        26 de Fevereiro & 2                                                         & Elaboração da proposta.                                          \\
        \hline
        12 de Março     & 2                                                         & Implementação das bases de dados local e remota.                 \\
        \hline
        26 de Março     & 2                                                         & Implementação do algoritmo goeBURST e scheduling.                \\
        \hline
        9 de Abril      & 2                                                         & Visualização através do algoritmo Force-Directed Layout.         \\
        \hline
        23 de Abril     & 1                                                         & Finalização do relatório de progresso e apresentação individual. \\
        \hline
        30 de Abril     & 2                                                         & Visualização através do algoritmo GrapeTree Layout.              \\
        \hline
        14 de Maio      & 2                                                         & Criação e desenvolvimento do cartaz e entrega da versão beta.    \\
        \hline
        28 de Maio      & 6                                                         & Otimizações, finalização do relatório e entrega da versão final. \\
        \hline
    \end{tabular}
    \caption{Calendarização do projeto.}
    \label{calendar}
\end{table}

Ao longo de todas as semanas serão também desenvolvidos testes unitários, documentação de código e o relatório.

%\bibliographystyle{abbrv}
%\bibliography{references}
%\end{document}

\begin{thebibliography}{9}


\end{thebibliography}

\end{document}
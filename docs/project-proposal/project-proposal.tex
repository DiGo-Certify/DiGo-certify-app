\documentclass[10pt]{article}

%% Language and font encodings
\usepackage[english]{babel}
\usepackage[utf8x]{inputenc}
\usepackage[T1]{fontenc}
\usepackage{url}

%% Sets page size and margins
\usepackage[a4paper, top=2cm, bottom=2cm, left=2cm, right=2cm, marginparwidth=1.75cm]{geometry}

%% Useful packages
\usepackage{graphicx}

\begin{document}
\begin{center}

\includegraphics[scale=0.5]{assets/logo.png}

\begin{LARGE}
\textbf{SCAR: A Blockchain based approach for academic registry}
\end{LARGE}

\vspace{.75cm} Projeto e Seminário

\vspace{.25cm} Licenciatura em Engenharia Informática e Computadores

\vspace{.75cm}
\begin{large}
\begin{tabular}{ c c c }
Diogo Rodrigues & Gonçalo Frutuoso \\
\texttt{49513@alunos.isel.pt} & \texttt{49495@alunos.isel.pt} \\
\end{tabular}

\vspace{.75cm} Orientadores \vspace{0.25cm}

\begin{tabular}{ c c }
Cátia Vaz & Alexandre Francisco \\
\texttt{cvaz@cc.isel.ipl.pt} & \texttt{aplf@tecnico.pt}
\end{tabular}
\end{large}

\vspace{.75cm} \today \vspace{.75cm}

\end{center}

\section{Introdução}



\section{Requisitos}


\section{Arquitetura}

\pagebreak

\section{Calendarização}

\begin{table}[!h]
\centering
\begin{tabular}{| c | c | c |}
\hline
Data de início & \begin{tabular}{@{}c@{}}Duração \\ (semanas)\end{tabular} & Descrição \\
\hline
19 de Fevereiro & 1 & Estudo e escolha dos componentes a usar. \\
\hline
19 de Fevereiro & 2 & Desenho da arquitetura da aplicação. \\
\hline
26 de Fevereiro & 2 & Elaboração da proposta. \\
\hline
12 de Março & 2 & Implementação das bases de dados local e remota. \\
\hline
26 de Março & 2 & Implementação do algoritmo goeBURST e scheduling. \\
\hline
9 de Abril & 2 & Visualização através do algoritmo Force-Directed Layout. \\
\hline
23 de Abril & 1 & Finalização do relatório de progresso e apresentação individual. \\
\hline
30 de Abril & 2 & Visualização através do algoritmo GrapeTree Layout. \\
\hline
14 de Maio & 2 & Criação e desenvolvimento do cartaz e entrega da versão beta. \\
\hline
28 de Maio & 6 & Otimizações, finalização do relatório e entrega da versão final. \\
\hline
\end{tabular}
\caption{Calendarização do projeto.}
\label{calendar}
\end{table}

Ao longo de todas as semanas serão também desenvolvidos testes unitários, documentação de código e o relatório.

%\bibliographystyle{abbrv}
%\bibliography{references}
%\end{document}

\begin{thebibliography}{9}


\end{thebibliography}

\end{document}
\documentclass[10pt]{article}

%% Language and font encodings
\usepackage[english]{babel}
\usepackage[utf8x]{inputenc}
\usepackage[T1]{fontenc}
\usepackage{url}

%% Sets page size and margins
\usepackage[a4paper, top=2cm, bottom=2cm, left=2cm, right=2cm, marginparwidth=1.75cm]{geometry}

%% Useful packages
\usepackage{graphicx}

\begin{document}
\begin{center}

    \includegraphics[scale=0.5]{assets/logo.png}

    \begin{LARGE}
        \textbf{SCAR\@: A \textit{blockchain} based approach for academic registry}
    \end{LARGE}

    \vspace{.75cm} Projeto e Seminário

    \vspace{.25cm} Licenciatura em Engenharia Informática e Computadores

    \vspace{.75cm}
    \begin{large}
        \begin{tabular}{ c c c }
            Diogo Rodrigues               & Gonçalo Frutuoso              \\
            \texttt{49513@alunos.isel.pt} & \texttt{49495@alunos.isel.pt} \\
        \end{tabular}

        \vspace{.75cm} Orientadores \vspace{0.25cm}

        \begin{tabular}{ c c }
            Cátia Vaz                    & Alexandre Francisco      \\
            \texttt{cvaz@cc.isel.ipl.pt} & \texttt{aplf@tecnico.pt}
        \end{tabular}
    \end{large}

    \vspace{.75cm} \today \vspace{.75cm}

\end{center}

\section{Introdução}

A veracidade e autenticidade dos registos académicos, especialmente certificados e certidões de notas,
têm sido um desafio persistente no campo da educação. A confiabilidade desses documentos é crucial não apenas
para os próprios indivíduos, mas também para instituições de ensino, empregadores e outras entidades que
dependem desses registos para tomar decisões importantes. No entanto, a atual gestão desses registos
ainda é amplamente baseada em métodos tradicionais, como impressões em papel ou documentos digitais simples,
o que os torna vulneráveis à adulteração, falsificação ou perda. A crescente incidência de fraudes
relacionadas a certificados académicos, juntamente com os desafios associados à validação manual desses
documentos, destaca a necessidade urgente de uma solução inovadora e segura.

Atualmente, embora existam algumas soluções digitais disponíveis para a gestão de registos académicos,
muitas delas carecem de uma abordagem verdadeiramente segura e confiável. Os sistemas de gestão académicos
(AMS~\cite{LinWays}) e os portais de estudantes geralmente oferecem opções para visualizar e descarregar  certificados e certidões
de notas, porém, esses documentos digitais ainda podem ser facilmente manipulados ou falsificados. Além disso,
a verificação desses registos é muitas vezes um processo manual e demorado, envolvendo a comunicação direta entre
as partes interessadas ou a consulta a bases de dados centralizadas, o que pode comprometer a eficiência
e a integridade do processo.

Diante desse cenário, surge a necessidade premente de uma solução que possa garantir a autenticidade e a
imutabilidade dos registos académicos, ao mesmo tempo em que simplifica e agiliza o processo de validação.
É nesse contexto que a proposta de projeto, denominada SCAR (\textit{Smart Contracts for Academic Registry}),
se destaca como uma abordagem inovadora e eficaz para a gestão de certificados e certidões de notas por
meio da tecnologia \textit{blockchain} e \textit{smart contracts}.

O principal objetivo do projeto SCAR é desenvolver uma aplicação móvel que utilize a tecnologia de \textit{smart contracts}
para garantir a autenticidade e a imutabilidade dos registos académicos, especialmente certificados e certidões
de notas, por meio da \textit{blockchain}. Além disso, o projeto visa simplificar o processo de validação desses documentos,
proporcionando uma solução segura e eficiente para alunos ou \textit{alumni}, empregadores ou entidades externas e administradores da faculdade.

\section*{Tecnologias}

Embora não tenha sido lecionado ao longo da licenciatura, considerou-se interessante o desafio de aprender
as tecnologias \texttt{React Native}~\cite{ReactNativeBook}, \texttt{Solidity}~\cite{SolidityBook} e \texttt{Expo}~\cite{Expo},
devido à sua capacidade de oferecer uma abordagem moderna e eficiente para a criação de uma aplicação móvel robusta e segura.
O \texttt{React Native} e o \texttt{Expo} são \textit{frameworks} conhecidas por facilitarem o desenvolvimento de aplicativos móveis multiplataforma, 
enquanto o \texttt{Solidity} desempenha um papel crucial na integração dos smart contracts na aplicação SCAR\@.
Além disso, é de salientar a \texttt{Web3}~\cite{Web3}, uma infraestrutura que promove a descentralização da internet, tornando-se interessante no contexto de
aplicações descentralizadas e \textit{blockchain}, podendo ser considerada para uma integração mais ampla com a aplicação SCAR\@.

\subsection*{\texttt{React Native}}

\texttt{React Native}~\cite{ReactNativeDocs}, juntamente com a plataforma \texttt{Expo}, emerge como uma das ferramentas mais versáteis
para o desenvolvimento de aplicações móveis. Ao unir a eficiência do \texttt{React.js} com a capacidade de compilação
nativa, \texttt{React Native} permite aos programadores criar aplicações móveis robustas e de alto desempenho
com uma base de código compartilhada. A plataforma \texttt{Expo}, por sua vez, oferece um conjunto de ferramentas
e serviços adicionais que simplificam e aceleram o processo de desenvolvimento, permitindo aos programadores
concentrarem-se na criação de experiências de utilizador excepcionais.

\subsection*{\texttt{Solidity}}

\texttt{Solidity}~\cite{SolidityDocs} é a linguagem de programação utilizada para escrever \textit{smart contracts} na \textit{blockchain} Ethereum~\cite{Ethereum}.
Os \textit{smart contracts}, representam a base da automatização e da fiabilidade na \textit{blockchain} e são essenciais
para garantir a execução transparente e imutável das transações e acordos entre as partes envolvidas.
À medida que a adoção da tecnologia \textit{blockchain} continua a crescer e a expandir-se para além do mundo das criptomoedas, o domínio de \texttt{Solidity} torna-se
uma habilidade altamente valorizada e procurada.\ \textit{Smart Contracts} desempenham um papel fundamental numa
ampla gama de aplicações distribuidas (\textit{dApps}), desde finanças descentralizadas (\textit{DeFi}) até votação eletrónica, entre outros.

\section{Requisitos}

A proposta de projeto SCAR tem como objetivo atender às necessidades de diferentes intervenientes no ecossistema educacional.
Os requisitos obrigatórios e opcionais foram definidos com base nas funcionalidades essenciais e nas melhorias adicionais
desejadas para a aplicação.

\subsection*{Intervenientes}

A solução SCAR envolve três principais intervenientes: os estudantes ou \textit{alumni}, os empregadores ou entidades terceiras e os administradores
de faculdade. Cada um desempenha um papel crucial no ecossistema da aplicação.

\subsection*{Papel de cada interveniente e ações que podem realizar}

\begin{itemize}

    \item \textbf{Estudantes ou \textit{alumni}}: Os estudantes ou alumni são a essência da plataforma SCAR.\@Terão a capacidade de se registarem na
          aplicação móvel, onde poderão carregar e armazenar os seus certificados académicos de forma segura na \textit{blockchain}.
          Além disso, os estudantes ou alumni poderão visualizar e compartilhar os seus certificados com empregadores ou entidades terceiras durante processos
          de recrutamento, proporcionando uma experiência transparente e eficiente.

    \item \textbf{Administradores de Faculdade:} Os administradores da faculdade são responsáveis por fornecer e validar
          os certificados académicos emitidos pela instituição de ensino. Terão a capacidade de aceder à aplicação
          para emitir e autenticar os certificados dos estudantes ou alumni, garantindo assim a integridade e a validade dos documentos.

    \item \textbf{Empregadores ou Entidades externas:} Os empregadores desempenham um papel crucial ao solicitar a verificação de certificados
          académicos durante entrevistas de emprego. Ao aceder a plataforma SCAR, os empregadores ou entidades externas podem verificar
          instantaneamente a autenticidade dos certificados apresentados pelos estudantes ou alumni, garantindo assim um processo
          de recrutamento mais confiável e transparente.

\end{itemize}

\subsection*{Requisitos obrigatórios}

\begin{itemize}

    \item\textbf{Implementação da aplicação móvel utilizando \texttt{React Native} com a plataforma \texttt{Expo}:} A escolha do \texttt{React Native}
          juntamente com a plataforma \texttt{Expo} proporciona uma abordagem eficaz para o desenvolvimento de aplicações móveis
          multiplataforma. Isso garante uma experiência consistente para os utilizadores, independentemente do dispositivo
          utilizado, e facilita o processo de desenvolvimento para a equipa de desenvolvimento.

    \item\textbf{Utilização de \textit{smart contracts}  em \texttt{Solidity} para armazenar e validar os certificados académicos na \textit{blockchain}
              Ethereum:} A utilização de \textit{smart contracts} em \texttt{Solidity} oferece uma solução segura e confiável para armazenar
          e validar os certificados académicos na \textit{blockchain} Ethereum. Essa abordagem garante a integridade e a
          imutabilidade dos dados, tornando a plataforma SCAR altamente confiável e resistente a fraudes.

    \item\textbf{Desenvolvimento de funcionalidades para registo, autenticação e armazenamento seguro de certificados na
              \textit{blockchain}:} O desenvolvimento de funcionalidades robustas para registro, autenticação e armazenamento seguro
          de certificados na \textit{blockchain} é essencial para garantir a segurança e a confiabilidade da plataforma SCAR.\@
          Essas funcionalidades devem ser projetadas com foco na usabilidade e na segurança, proporcionando uma
          experiência intuitiva e transparente para os utilizadores.

\end{itemize}

\subsection*{Requisitos opcionais}

\begin{itemize}

    \item\textbf{Integração de funcionalidades adicionais, como notificações em tempo real e partilha de certificados
              através de meios digitais:} A integração de funcionalidades adicionais, como notificações em tempo real e
          compartilhamento de certificados através de meios digitais como por exemplo redes sociais, pode melhorar ainda mais a experiência do
          utilizador na plataforma SCAR, aumentando a usabilidade e adesão de novos utilizadores.

\end{itemize}

\pagebreak

\section{Calendarização}

\begin{table}[!h]
    \centering
    \begin{tabular}{| c | c | c |}
        \hline
        Data de início  & \begin{tabular}{@{}c@{}}Duração \\ (semanas)\end{tabular} & Descrição                                                       \\
        \hline
        16 de Fevereiro & 2                                                         & Estudo sobre a framework de \textit{frontend} a utilizar,       \\ & & \textbf{Expo \texttt{React Native}}                  \\
        \hline
        28 de Fevereiro & 2                                                         & Estudo sobre a tecnologia \textit{Smart Contracts}              \\ & & e a linguagem que esta suporta, \textbf{\texttt{Solidity}}        \\
        \hline
        6 de Março      & 2                                                         & Elaboração da proposta                                          \\
        \hline
        18 de Março     & 2                                                         & Decisão acerca da estrutura e modelo relacional a utilizar      \\
        \hline
        29 de Março     & 1                                                         & Elaboração do mockup do \textit{frontEnd} da aplicação          \\
        \hline
        5 de Abril      & 2                                                         & Implementação dos ecrãs da aplicação                            \\
        \hline
        5 de Abril      & 3                                                         & Implementação dos \textbf{Smart Contracts} necessários          \\
        \hline
        15 de Abril     & 1                                                         & Apresentação individual                                         \\
        \hline
        30 de Abril     & 2                                                         & Testes, avaliação experimental incluindo testes de usabilidade  \\
        \hline
        13 de Maio      & 2                                                         & Criação e desenvolvimento do cartaz e entrega da versão beta    \\
        \hline
        28 de Maio      & 6                                                         & Otimizações, finalização do relatório e entrega da versão final \\
        \hline
    \end{tabular}
    \caption{Calendarização do projeto.}\label{calendar}
\end{table}

Ao longo do trabalho efetuado semanalmente, irá ser elaborado continuadamente documentação, testes unitários e relatório do projeto, de forma a que no final do mesmo, o relatório esteja completo e pronto a ser entregue.

\vspace{.75cm}

\bibliography{references}
\bibliographystyle{abbrv}

\end{document}
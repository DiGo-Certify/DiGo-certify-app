%
% Resumo
%
\chapter*{Resumo}\label{chap:resumo}

Um grande desafio na era digital dos dias de hoje é verificar a autenticidade de certificados académicos. Entre as questões significativas com que se depara o sector da educação contam-se o problema generalizado de certificados falsificados e a ausência de um sistema fiável e global para confirmar os resultados académicos.
O processo de verificação é muitas vezes complicado devido aos sistemas tradicionais, que dependem de instituições de ensino específicas.
Este facto pode levantar questões quanto à autenticidade dos certificados académicos. Os trabalhadores também têm de atualizar e validar as suas capacidades regularmente, mas não existem locais fiáveis suficientes para mostrar essas competências.

Nesta investigação, é apresentada uma solução inovadora que utiliza a tecnologia \textit{blockchain} para resolver estas dificuldades: a aplicação DiGo Certify. Esta aplicação móvel procura garantir procedimentos à prova de fraude, utilizando a tecnologia de \texttt{Smart Contracts}~\cite{vigliotti2021we} para automatizar e garantir a validação e emissão de certificados académicos.

Concebido para cumprir o Regulamento Geral sobre a Proteção de Dados (RGPD), este sistema permite que os indivíduos tenham controlo sobre os seus dados pessoais, mantendo a integridade e a transparência dos registos académicos.

A aplicação DiGo Certify integra-se facilmente nos sistemas institucionais existentes, evitando a necessidade de atualizações significativas, sendo suficientemente versátil para certificar vários tipos de informação académica, abrangendo tanto as qualificações formais como a aprendizagem informal.
Este documento examina as vantagens e implicações deste modelo, fazendo comparação com outros tipos de modelos e enfatizando sempre o papel crucial da privacidade e da proteção de dados em soluções educativas baseadas em blockchain.

\paragraph{Palavras-chave:} Verificação de Certificados Académicos, Tecnologia Blockchain, Smart Contracts, Aplicação DiGo Certify, Prevenção de Fraude, Conformidade RGPD, Registos Académicos, Privacidade de Dados, Certificação Académica, Soluções Educativas Digitais
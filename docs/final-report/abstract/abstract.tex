%
% Abstract
% 
\chapter*{Abstract}\label{chap:abstract}

A major difficulty in the digital age of today is verifying the legitimacy of academic credentials. Significant issues facing the education sector include the pervasive problem of phony credentials and the absence of a reliable, global system for confirming academic accomplishments. The verification procedure is often complicated by traditional systems, which are dependent on specific educational institutions. This might raise questions regarding the authenticity of academic records. Professionals must also refresh and validate their abilities on a regular basis, but there aren't enough trustworthy venues to display these competencies.

In this research, an innovative solution utilizing blockchain technology to address these difficulties is presented: the DiGo Certify App. This App ensures fraud-proof procedures by utilizing smart contracts to automate and secure academic credential validation and issuing.

Designed to comply with the General Data Protection Regulation (GDPR), this system empowers individuals with control over their personal data while maintaining the integrity and transparency of academic records.

The DiGo Certify App integrates smoothly with existing institutional systems, avoiding the need for significant upgrades. It is versatile enough to certify various types of academic information, covering both formal qualifications and informal learning. This work examines the advantages and implications of this model, emphasizing the crucial role of privacy and data protection in blockchain-based educational solutions.

\paragraph{Keywords:} Academic Credential Verification, Blockchain Technology, Smart Contracts, DiGo Certify App, Fraud Prevention, GDPR Compliance, Educational Records, Data Privacy, Academic Certification, Digital Education Solutions
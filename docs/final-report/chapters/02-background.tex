%
% Chapter 2
%
\chapter{Background}\label{chap:background}

% Do an introduction to the chapter

\section{Introduction to the Problem}\label{sec:introduction-to-the-problem}
\paragraph{}

In today's fast paced world, the authenticity and accessibility of academic certificates play a crucial role in ensuring trust and credibility in various
domains, ranging from education to employment and beyond.
The current and traditional \textit{paper-based} system of issuing and verifying academic certificates is not only time consuming but also prone to a lot of fraud and manipulation.
The rampant proliferation of counterfeit certificates, inefficient verification processes and the risk of loss or damage highlight the need for a more reliable, robust and secure academic
certificate registry system.

The current system of academic certificate registry is plagued by numerous challenges. Firstly, the reliance and trust on paper-based certificates is a major issue
making them susceptible to forgery and tampering undermining the cridibility and integrity of academic qualifications. Secondly, the manual verification process is
time-consuming and prone to errors, leading to delays in credential validation, possible fraudulent activities and also potential loss of revenue for institutions due to
errors in the manual release. Thirdly, the centralized nature of certificate issuance by educational intituitions exacerbates the difficulty
of maintaining a unified and updated registry, hampering efficient verification mechanisms.

% based on some articles 
\section{Alternative Approaches}\label{sec:alternative-approaches}
\paragraph{}

Several attemps have been made to address the imperfections of the traditional academic certificate registry system.
One such solution is the implementation of \textit{centralized databases}~\cite{OLSON200971} managed by government or regulatory authorities, where educational institutions are required to
submit digital copies of certificates for verification purposes. Additionally this approach aims to centralize certificate records and simplify the verification process, it still faces challenges
such as the risk and concerns of data privacy and security, interoperability issues between different databases and the need of a trusted third party to manage the database.
This centralized mechanism of keeping record is also devoted to have a single point of failure.
% missing examples of centralized databases

Another solution that is gaining traction is the adoption of \textit{blockchain technology} for academic certificate registry. Blockchain offers a decentralized, secure and tamper-proof ledger where certificates can be stored and verified.
The use of blockchain technology ensures that certificates are immutable, transparent and accessible to all stakeholders. Moreover, this technology enables the instant verification trough cryptographic methods,
eliminating the need for a central authority to manage the registry, thereby reducing the risk of fraud and manipulation.
This approach eliminates the need for a central authority to manage the registry, thereby reducing the risk of fraud and manipulation.

In contrast to the traditional centralized databased system, in our opinion, blockchain emerges as a disruptive force capable of revolutionizing academic certificate registry systems
by providing in a decentralized and secure manner, an immutable and tamper-proof ledger where certificates will be stored and verified.
The decision to embrace blockchain technology as the foundation of our solution is based on what we sad above as well as the fact that blockchain technology is a key enabler of the \textit{Web3} vision,
which aims to create decentralized applications (dApps) that are secure, transparent and trustless where users have full control over their data and digital assets without having a \textbf{single point of failure}.
% missing examples of blockchain technology

\pagebreak % temporary solution to avoid the section title to be alone in the previous page

% tempo de duração de uma transação
\section{Blockchain}\label{sec:blockchain}
\paragraph{}

\section{Differente Approaches to Blockchain}\label{sec:different-approaches-to-blockchain}
\paragraph{}





%
% Chapter 1
%
\chapter{Introduction}\label{chap:introduction}

With the advent of the digital era, the management of academic records presents substantial challenges. The prevalence of forged certificates and the absence of a universally trusted system for issuing and verifying academic credentials contribute to significant issues within the education sector. Traditionally, academic records are managed by individual institutions through their proprietary systems, making verification by external stakeholders cumbersome. This lack of transparency and accessibility complicates the process of confirming the authenticity of academic accomplishments, leading to potential mistrust and inefficiencies.

The inability to effectively verify academic credentials has facilitated the rise of degree mills~\cite{saleh2020blockchain}, which produce counterfeit certifications. These fraudulent activities undermine the credibility of genuine academic institutions and compromise the value of legitimate degrees~\cite{muzammil2010corrupt}. Furthermore, the current systems are susceptible to unauthorized alterations and inaccuracies, further eroding trust in academic records.

To address these challenges, it is essential to explore and compare the different types of academic certificate storage and verification systems. The first type is the traditional database system~\cite{OLSON200971}, presented in Section~\ref{sec:centralized-systems}, where records are stored in centralized databases managed by individual institutions.
While this method provides control to institutions, it often has a gap in the necessary security and accessibility features to prevent fraud and ensure efficient verification.

The second type of solution is presented in Section~\ref{sec:distributed-systems} and involves distributed systems, which use multiple servers to manage records across various nodes.
This approach enhances security and reduces the risk of data loss but still relies on trusted intermediaries, which can introduce vulnerabilities and inefficiencies.

The third and most innovative solution is the fully distributed system, described in Section~\ref{sec:blockchain} using \textit{blockchain technology}. Blockchain~\cite{saleh2020blockchain} offers a decentralized and tamper-proof method for recording and verifying academic credentials.
By leveraging smart contracts, academic records can be securely issued, stored, and verified on a blockchain. This approach ensures transparency, reduces the risk of fraud, and simplifies the verification process for stakeholders.

The SCAR project represents an innovative application of blockchain technology in the education sector. This system aims to streamline the issuance and verification of academic certificates, ensuring that all records are securely stored and easily accessible for verification. The use of smart contracts in the DiGo Certify App automates the process of certificate issuance and validation, making it efficient and tamper-proof.

Moreover, compliance with data protection regulations, such as the General Data Protection Regulation (GDPR), is a critical aspect of this solution. The DiGo Certify App is designed to adhere to GDPR principles, ensuring that personal data is protected while maintaining the integrity and transparency of academic records on the blockchain.

Through the SCAR project and the DiGo Certify App, we aim to demonstrate how blockchain technology can revolutionize academic credential management, providing a secure and transparent system for the future.



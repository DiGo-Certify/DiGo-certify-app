%
%
%
\chapter{Testing}\label{chap:testing}

In this chapter, we present the testing processes employed for the application code. We
discuss both manual and automatic testing approaches.

\section{Code Testing}

In order to ensure the quality of the code, we have employed a number of tests to verify the correctness of the code. We have only automatic testing approaches for the blockchain code.
Manual testing is described in Section~\ref{chap:future-work}.

\subsection{Automatic Testing}

7.1.2 Automatic Testing
Automatic testing involves the use of software tools to verify the correctness and robustness of the application. This approach enhances the quality of the code and helps to identify bugs early in the development process. We have employed the following automatic testing approaches for the blockchain code:

\begin{itemize}
    \item \textbf{Unit Testing:} We have used the \textbf{chai} testing framework to write unit tests for the blockchain code. The unit tests verify the correctness of individual functions and methods in the code. We have written unit tests for all the functions and methods in the ethereum service code to ensure that they work as expected.

    \item \textbf{Integration Testing:} At this stage, we have not written integration tests for the blockchain code. Integration tests verify the interaction between different components of the application. We plan to write integration tests for the blockchain code in the future.

\end{itemize}

By implementing automated testing, we improved the application's reliability and maintainability while also speeding up the development process.





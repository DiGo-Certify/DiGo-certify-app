%
% Chapter 3
%
\chapter{Requirements}\label{chap:requirements}

With the purpose of fulfilling all the objectives of the project SCAR, we have developed DiGo Certify mobile application. Our mobile application, allows different interveners to access, share and emit certificates to the Ethereum blockchain, from anywhere in the world, offering a futuristic solution for the problems described in section (…).

The SCAR project proposal aims to meet the needs of different stakeholders in the educational ecosystem. Mandatory and optional requirements have been defined based on essential functionalities and additional enhancements desired for the application.

TODO(FINISH REQUIRMENTS INTRO)

\subsection{Stakeholders}\label{subsec:stakeholders}
The SCAR solution involves three main stakeholders: students or alumni, employers or third-party entities, and college administrators. Each plays a crucial role in the application's ecosystem.

\subsubsection{Role of Each Stakeholder and Actions They Can Perform:}

\begin{itemize}

    \item \textbf{Students or Alumni:} Students or alumni are the essence of the SCAR platform. They will have the ability to register on the mobile application, where they can securely upload and store their academic certificates on the blockchain. Additionally, students or alumni can view and share their certificates with employers or third-party entities during recruitment processes, providing a transparent and efficient experience.

    \item \textbf{College Administrators:} College administrators are responsible for providing and validating the academic certificates issued by the educational institution. They will have the ability to access the application to issue and authenticate students' or alumni's certificates, thus ensuring the integrity and validity of the documents.

    \item \textbf{Employers or External Entities:} Employers play a crucial role in requesting the verification of academic certificates during job interviews. By accessing the SCAR platform, employers or external entities can instantly verify the authenticity of the certificates presented by students or alumni, without the need of possessing a wallet and with this ensuring a more reliable and transparent recruitment process.

\end{itemize}

\subsection{Mandatory Requirements:}

\begin{itemize}

    \item \textbf{Implementation of the mobile application using React Native with Expo platform:} The choice of React Native along with the Expo platform provides an effective approach for developing multiplatform mobile applications. This ensures a consistent experience for users, regardless of the device used, and facilitates the development process for the development team.

    \item \textbf{Utilization of smart contracts in Solidity to store and validate academic certificates on the Ethereum blockchain:} The use of smart contracts in Solidity offers a secure and reliable solution for storing and validating  academic certificates on the Ethereum blockchain. This approach ensures data integrity and immutability, making the SCAR platform highly reliable and resistant to fraud.

    \item \textbf{Development of features for registration, authentication, and secure storage of certificates on the blockchain:} Developing robust features for registering, authenticating, and securely storing certificates on the blockchain is essential to ensure the security and reliability of the SCAR platform. These features should be designed with a focus on usability and security, providing an intuitive and transparent experience for users.

\end{itemize}

\subsection{Optional Requirements:}

\begin{itemize}

    \item \textbf{Integration of additional features, such as real-time notifications and sharing of certificates through digital channels:} Incorporating extra functionalities like real-time notifications and certificate sharing through digital channels, such as social media platforms, can further enhance the user experience on the SCAR platform. This enhancement can boost usability and attract new users to the platform.
    
\end{itemize}

\section{Use Cases}\label{sec:use-cases}

